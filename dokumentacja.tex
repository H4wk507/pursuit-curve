\documentclass[12pt,a4paper]{article}

\usepackage[utf8]{inputenc}
\usepackage[polish]{babel}
\usepackage[T1]{fontenc}
\usepackage{amsmath}
\usepackage{amsfonts}
\usepackage{graphicx}
\usepackage{geometry}
\usepackage{hyperref}
\usepackage{float}
\usepackage{listings}
\usepackage{xcolor}
\usepackage{booktabs}

\geometry{
    left=2.5cm,
    right=2.5cm,
    top=2.5cm,
    bottom=2.5cm
}

\lstset{
    basicstyle=\ttfamily\small,
    breaklines=true,
    frame=single,
    commentstyle=\color{gray},
    keywordstyle=\color{blue},
    stringstyle=\color{red}
}

\begin{document}

\begin{flushright}
Gliwice, \today
\end{flushright}

\vspace{1cm}

\begin{center}
{\Large \textbf{Modelowanie matematyczne}}\\[0.5cm]
{\Large Dokumentacja projektu końcowego}\\[1cm]

{\large \textbf{Tytuł:} Krzywa pogoni}\\[0.5cm]

{\large \textbf{Autorzy:} Skowroński Piotr, Czuba Krzysztof}\\[0.3cm]

{\large \textbf{Kierunek:} Informatyka, studia 2 stopnia (sem. II)}
\end{center}

\vspace{1cm}

\tableofcontents
\newpage

\section{Cel zadania}

Celem projektu jest implementacja i analiza różnych strategii pościgu (pursuit curves) w przestrzeni euklidesowej oraz na rozmaitościach riemannowskich. Projekt obejmuje:

\begin{itemize}
    \item Implementację klasycznych strategii pościgu w 2D: Direct Pursuit, Constant Bearing, Proportional Navigation, Cyclic Pursuit
    \item Rozszerzenie strategii na przestrzeń 3D oraz ogólną przestrzeń n-wymiarową
    \item Implementację pościgu na rozmaitościach riemannowskich: sfera $S^2$ oraz torus
    \item Rozwiązanie numeryczne równań różniczkowych opisujących ruch ścigającego i celu
    \item Analizę zachowania strategii dla różnych trajektorii celu (linia prosta, okrąg, helisa, krzywe Lissajous)
    \item Porównanie podejść ciągłych (ODE) i dyskretnych (iteracyjnych)
    \item Wizualizację trajektorii w 2D i 3D oraz analizę numeryczną dla wyższych wymiarów
\end{itemize}

\section{Opis teoretyczny}

\subsection{Wprowadzenie do problemu pościgu}

Problem pościgu (pursuit problem) polega na określeniu trajektorii ścigającego (pursuer), który dąży do przechwycenia poruszającego się celu (target). W zależności od zastosowanej strategii, ścigający może celować bezpośrednio w aktualną pozycję celu (Direct Pursuit), utrzymywać stały kąt namiarowania (Constant Bearing), lub stosować bardziej zaawansowane metody jak nawigacja proporcjonalna (Proportional Navigation).

Problemy pościgu mają szerokie zastosowania praktyczne:
\begin{itemize}
    \item Nawigacja rakiet i pocisków naprowadzanych
    \item Systemy autonomiczne (drony, roboty)
    \item Nawigacja morska i lotnicza
    \item Gry komputerowe (AI przeciwników)
    \item Optymalizacja trajektorii
\end{itemize}

\subsection{Klasyczna krzywa pościgu (Direct Pursuit)}

\subsubsection{Model matematyczny 2D}

W najprostszym przypadku Direct Pursuit, ścigający w każdej chwili celuje bezpośrednio w aktualną pozycję celu. Stan układu w przestrzeni dwuwymiarowej $\mathbb{R}^2$ opisany jest wektorami:

\begin{itemize}
    \item $\mathbf{p}(t) = [p_x(t), p_y(t)]^T$ -- pozycja ścigającego
    \item $\mathbf{t}(t) = [t_x(t), t_y(t)]^T$ -- pozycja celu
\end{itemize}

Równania różniczkowe opisujące ruch ścigającego w strategii Direct Pursuit mają postać:

\begin{equation}
\frac{d\mathbf{p}}{dt} = v_p \cdot \frac{\mathbf{t} - \mathbf{p}}{\|\mathbf{t} - \mathbf{p}\|}
\label{eq:direct_pursuit_2d}
\end{equation}

gdzie:
\begin{itemize}
    \item $v_p > 0$ -- prędkość skalarna ścigającego [m/s]
    \item $\mathbf{t} - \mathbf{p}$ -- wektor skierowany od ścigającego do celu
    \item $\|\mathbf{t} - \mathbf{p}\| = \sqrt{(t_x - p_x)^2 + (t_y - p_y)^2}$ -- odległość euklidesowa
\end{itemize}

Ruch celu opisany jest niezależnym równaniem:
\begin{equation}
\frac{d\mathbf{t}}{dt} = \mathbf{v}_t(t)
\label{eq:target_motion}
\end{equation}

gdzie $\mathbf{v}_t(t)$ jest zadaną funkcją prędkości celu.

\subsubsection{Uogólnienie na przestrzeń 3D}

W przestrzeni trójwymiarowej $\mathbb{R}^3$, równania pościgu bezpośredniego mają analogiczną postać:

\begin{equation}
\frac{d\mathbf{p}}{dt} = v_p \cdot \frac{\mathbf{t} - \mathbf{p}}{\|\mathbf{t} - \mathbf{p}\|}
\end{equation}

gdzie $\mathbf{p}(t) = [p_x(t), p_y(t), p_z(t)]^T$ i $\mathbf{t}(t) = [t_x(t), t_y(t), t_z(t)]^T$, a odległość euklidesowa:
\begin{equation}
\|\mathbf{t} - \mathbf{p}\| = \sqrt{(t_x - p_x)^2 + (t_y - p_y)^2 + (t_z - p_z)^2}
\end{equation}

\subsubsection{Przestrzeń n-wymiarowa}

Dla ogólnej przestrzeni $\mathbb{R}^n$, strategia Direct Pursuit generalizuje się do:

\begin{equation}
\frac{d\mathbf{p}}{dt} = v_p \cdot \frac{\mathbf{t} - \mathbf{p}}{\|\mathbf{t} - \mathbf{p}\|_2}
\end{equation}

gdzie $\mathbf{p}, \mathbf{t} \in \mathbb{R}^n$ oraz $\|\cdot\|_2$ oznacza normę euklidesową:
\begin{equation}
\|\mathbf{x}\|_2 = \sqrt{\sum_{i=1}^{n} x_i^2}
\end{equation}

\subsection{Constant Bearing (Stały kąt namiarowania)}

W strategii Constant Bearing, ścigający porusza się pod stałym kątem $\beta$ względem kierunku do celu. Prowadzi to do trajektorii w postaci spirali logarytmicznej.

\subsubsection{Model matematyczny 2D}

Kierunek ruchu ścigającego:
\begin{equation}
\theta_{\text{target}} = \arctan2(t_y - p_y, t_x - p_x)
\end{equation}

\begin{equation}
\theta_{\text{pursuer}} = \theta_{\text{target}} + \beta
\end{equation}

Równanie ruchu:
\begin{equation}
\frac{d\mathbf{p}}{dt} = v_p \cdot [\cos(\theta_{\text{pursuer}}), \sin(\theta_{\text{pursuer}})]^T
\end{equation}

gdzie $\beta$ jest stałym kątem namiarowania (parametr strategii).

\subsection{Proportional Navigation (Nawigacja proporcjonalna)}

Strategia Proportional Navigation jest szeroko stosowana w naprowadzaniu pocisków. Prędkość kątowa ścigającego jest proporcjonalna do prędkości kątowej linii wzroku (Line-Of-Sight, LOS).

\subsubsection{Model matematyczny 2D}

Kąt linii wzroku:
\begin{equation}
\lambda(t) = \arctan2(t_y - p_y, t_x - p_x)
\end{equation}

Prędkość kątowa LOS:
\begin{equation}
\dot{\lambda}(t) = \frac{d\lambda}{dt}
\end{equation}

Nowy kąt ruchu ścigającego:
\begin{equation}
\theta_{\text{new}} = \theta_{\text{current}} + N \cdot \Delta\lambda
\end{equation}

gdzie:
\begin{itemize}
    \item $N$ -- stała nawigacji (typowo $N \in [3, 5]$)
    \item $\Delta\lambda$ -- zmiana kąta LOS w ostatnim kroku czasowym
\end{itemize}

Równanie ruchu:
\begin{equation}
\frac{d\mathbf{p}}{dt} = v_p \cdot [\cos(\theta_{\text{new}}), \sin(\theta_{\text{new}})]^T
\end{equation}

\subsection{Cyclic Pursuit (Pościg cykliczny)}

W pościgu cyklicznym, $n$ agentów jest ułożonych w konfiguracji początkowej (zazwyczaj okrąg), a każdy agent ściga kolejnego agenta według strategii Direct Pursuit. Prowadzi to do zbieżności wszystkich agentów do wspólnego punktu centralnego.

\subsubsection{Model matematyczny}

Dla $n$ agentów, stan układu składa się z $2n$ współrzędnych:
\begin{equation}
\mathbf{s}(t) = [x_1, y_1, x_2, y_2, \ldots, x_n, y_n]^T
\end{equation}

Równanie ruchu agenta $i$ ($i = 1, 2, \ldots, n$):
\begin{equation}
\frac{d\mathbf{p}_i}{dt} = v_i \cdot \frac{\mathbf{p}_{i+1} - \mathbf{p}_i}{\|\mathbf{p}_{i+1} - \mathbf{p}_i\|}
\end{equation}

gdzie $\mathbf{p}_{n+1} = \mathbf{p}_1$ (cykliczne indeksowanie).

\subsection{Trajektorie celu}

\subsubsection{Ruch liniowy}

Cel porusza się ze stałą prędkością:
\begin{equation}
\mathbf{v}_t(t) = \mathbf{v}_0 = \text{const}
\end{equation}

\subsubsection{Ruch po okręgu (2D)}

Cel wykonuje ruch jednostajny po okręgu o promieniu $r$ z prędkością kątową $\omega$:
\begin{equation}
\mathbf{t}(t) = [r\cos(\omega t), r\sin(\omega t)]^T
\end{equation}

\subsubsection{Helisa (3D)}

Trajektoria helisoidalna w przestrzeni 3D:
\begin{equation}
\mathbf{t}(t) = [r\cos(\omega t), r\sin(\omega t), v_z \cdot t]^T
\end{equation}

gdzie:
\begin{itemize}
    \item $r$ -- promień helisy
    \item $\omega$ -- prędkość kątowa
    \item $v_z$ -- prędkość pionowa
\end{itemize}

\subsubsection{Krzywe Lissajous (3D)}

Trajektoria opisana przez niezależne oscylacje harmoniczne w każdej współrzędnej:
\begin{equation}
\mathbf{t}(t) = [A_x \sin(\omega_x t), A_y \sin(\omega_y t), A_z \sin(\omega_z t)]^T
\end{equation}

gdzie:
\begin{itemize}
    \item $A_x, A_y, A_z$ -- amplitudy oscylacji
    \item $\omega_x, \omega_y, \omega_z$ -- częstotliwości kątowe
\end{itemize}

\subsection{Pościg na rozmaitościach riemannowskich}

\subsubsection{Pościg na sferze $S^2$}

Dla pościgu na powierzchni sfery o promieniu $R$, pozycje są parametryzowane współrzędnymi sferycznymi $(r, \theta, \phi)$:
\begin{itemize}
    \item $r$ -- promień (stały, $r = R$)
    \item $\theta$ -- kąt polarny (colatitude), $\theta \in [0, \pi]$
    \item $\phi$ -- kąt azymutalny (longitude), $\phi \in [0, 2\pi)$
\end{itemize}

Konwersja do współrzędnych kartezjańskich:
\begin{equation}
\begin{bmatrix} x \\ y \\ z \end{bmatrix} = \begin{bmatrix} R \sin\theta \cos\phi \\ R \sin\theta \sin\phi \\ R \cos\theta \end{bmatrix}
\end{equation}

Metryka riemannowska na sferze:
\begin{equation}
ds^2 = R^2 d\theta^2 + R^2 \sin^2\theta \, d\phi^2
\end{equation}

Kierunek ruchu ścigającego jest rzutowany na płaszczyznę styczną do sfery, zapewniając ruch geodezyjny.

Odległość kątowa między dwoma punktami na sferze:
\begin{equation}
d_{\text{angular}} = \arccos(\cos\theta_p \cos\theta_t + \sin\theta_p \sin\theta_t \cos(\phi_t - \phi_p))
\end{equation}

\subsubsection{Pościg na torusie}

Torus jest parametryzowany dwoma kątami $(u, v)$:
\begin{itemize}
    \item $u \in [0, 2\pi)$ -- kąt wokół głównej osi
    \item $v \in [0, 2\pi)$ -- kąt wokół przekroju poprzecznego
\end{itemize}

Konwersja do współrzędnych kartezjańskich ($R$ -- promień główny, $r$ -- promień przekroju):
\begin{equation}
\begin{bmatrix} x \\ y \\ z \end{bmatrix} = \begin{bmatrix} (R + r\cos v) \cos u \\ (R + r\cos v) \sin u \\ r \sin v \end{bmatrix}
\end{equation}

Metryka riemannowska na torusie:
\begin{equation}
ds^2 = (R + r\cos v)^2 du^2 + r^2 dv^2
\end{equation}

Odległość riemannowska obliczana jest z uwzględnieniem metryki oraz okresowości kątów (warunki brzegowe toroidalne).

\subsection{Podsumowanie zaimplementowanych algorytmów}

Tabela \ref{tab:algorithms} przedstawia wszystkie zaimplementowane algorytmy pościgu.

\begin{table}[H]
\centering
\caption{Zaimplementowane algorytmy pościgu}
\label{tab:algorithms}
\begin{tabular}{@{}llll@{}}
\toprule
\textbf{Algorytm} & \textbf{Wymiar} & \textbf{Typ} & \textbf{Moduł} \\ \midrule
Direct Pursuit & 2D & Ciągły & \texttt{d2.continuous} \\
Constant Bearing & 2D & Ciągły & \texttt{d2.continuous} \\
Proportional Navigation & 2D & Ciągły & \texttt{d2.continuous} \\
Cyclic Pursuit & 2D & Ciągły & \texttt{d2.continuous} \\
Direct Pursuit & 2D & Dyskretny & \texttt{d2.discrete} \\
Constant Bearing & 2D & Dyskretny & \texttt{d2.discrete} \\
Proportional Navigation & 2D & Dyskretny & \texttt{d2.discrete} \\
Direct Pursuit & 3D & Ciągły & \texttt{d3.continuous} \\
Direct Pursuit & n-wymiarowy & Ciągły & \texttt{dn.continuous} \\
Direct Pursuit & Sfera $S^2$ & Ciągły & \texttt{sphere.continuous} \\
Direct Pursuit & Torus & Ciągły & \texttt{torus.continuous} \\ \bottomrule
\end{tabular}
\end{table}

\subsection{Trajektorie celu -- implementacje}

Tabela \ref{tab:targets} przedstawia wszystkie zaimplementowane trajektorie celu.

\begin{table}[H]
\centering
\caption{Zaimplementowane trajektorie celu}
\label{tab:targets}
\begin{tabular}{@{}lll@{}}
\toprule
\textbf{Trajektoria} & \textbf{Wymiar} & \textbf{Moduł} \\ \midrule
Ruch liniowy & 2D & \texttt{d2.continuous} \\
Ruch po okręgu & 2D & \texttt{d2.continuous} \\
Ruch liniowy & 3D & \texttt{d3.continuous} \\
Helisa & 3D & \texttt{d3.continuous} \\
Krzywe Lissajous & 3D & \texttt{d3.continuous} \\
Ruch liniowy & n-wymiarowy & \texttt{dn.continuous} \\
Ruch na sferze & Sfera $S^2$ & \texttt{sphere.continuous} \\
Ruch na torusie & Torus & \texttt{torus.continuous} \\ \bottomrule
\end{tabular}
\end{table}

\section{Implementacja}

\subsection{Architektura projektu}

Projekt zaimplementowany jest w języku Python 3.12+ z wykorzystaniem następujących bibliotek:
\begin{itemize}
    \item \texttt{numpy} -- operacje na tablicach i wektoryzacja obliczeń
    \item \texttt{scipy} -- rozwiązywanie równań różniczkowych (\texttt{solve\_ivp})
    \item \texttt{matplotlib} -- wizualizacja 2D i animacje
    \item \texttt{plotly} -- wizualizacja 3D interaktywna
\end{itemize}

\subsection{Struktura modułów}

\begin{itemize}
    \item \texttt{common/} -- klasy bazowe i wspólne narzędzia
    \begin{itemize}
        \item \texttt{types.py} -- definicje \texttt{Point2D}, \texttt{Point3D}, \texttt{PointND}, \texttt{Strategy}, \texttt{TargetStrategy}
        \item \texttt{continuous\_simulation.py} -- symulacja ciągła z użyciem \texttt{solve\_ivp}
    \end{itemize}
    \item \texttt{d2/} -- implementacje 2D (ciągłe i dyskretne)
    \item \texttt{d3/} -- implementacje 3D (ciągłe)
    \item \texttt{dn/} -- implementacje n-wymiarowe (ciągłe)
    \item \texttt{sphere/} -- pościg na sferze $S^2$
    \item \texttt{torus/} -- pościg na torusie
    \item \texttt{examples/} -- przykładowe skrypty demonstracyjne
\end{itemize}

\subsection{Podejście ciągłe vs. dyskretne}

\subsubsection{Symulacja ciągła}

W podejściu ciągłym, równania różniczkowe są rozwiązywane numerycznie za pomocą metody Runge-Kutty (implementacja \texttt{scipy.integrate.solve\_ivp}). Metoda ta zapewnia:
\begin{itemize}
    \item Adaptacyjny krok czasowy
    \item Kontrolę błędu (tolerancje względne i bezwzględne)
    \item Detekcję zdarzeń (warunki terminalne, np. przechwycenie celu)
    \item Gęste wyjście (interpolacja trajektorii)
\end{itemize}

Funkcja \texttt{run\_continuous\_simulation()} przyjmuje:
\begin{itemize}
    \item Strategię pościgu (\texttt{Strategy})
    \item Strategię ruchu celu (\texttt{TargetStrategy})
    \item Warunki początkowe
    \item Parametry symulacji (czas końcowy, tolerancje)
\end{itemize}

\subsubsection{Symulacja dyskretna}

W podejściu dyskretnym, równania są iterowane z ustalonym krokiem czasowym $\Delta t$:
\begin{equation}
\mathbf{p}_{k+1} = \mathbf{p}_k + \Delta t \cdot \mathbf{v}_p(\mathbf{p}_k, \mathbf{t}_k)
\end{equation}

\begin{equation}
\mathbf{t}_{k+1} = \mathbf{t}_k + \Delta t \cdot \mathbf{v}_t(\mathbf{t}_k)
\end{equation}

Podejście dyskretne jest prostsze w implementacji, ale może wprowadzać większe błędy numeryczne przy niewłaściwym doborze $\Delta t$.

\section{Wyniki eksperymentów}

\subsection{Pościg 2D -- Direct Pursuit}

\subsubsection{Cel nieruchomy}

\textbf{TODO:} Analiza numeryczna dla celu nieruchomego:
\begin{itemize}
    \item Porównanie czasu przechwycenia dla różnych prędkości ścigającego
    \item Długość trajektorii w funkcji pozycji początkowej
    \item Wykres trajektorii
\end{itemize}

\subsubsection{Cel poruszający się po okręgu}

\textbf{TODO:} Analiza numeryczna dla celu poruszającego się po okręgu:
\begin{itemize}
    \item Wpływ prędkości kątowej celu $\omega$ na kształt trajektorii
    \item Warunek konieczny do przechwycenia: $v_p > v_t$
    \item Wykres trajektorii dla różnych $\omega$
\end{itemize}

\subsection{Pościg 2D -- Constant Bearing}

\textbf{TODO:} Analiza numeryczna:
\begin{itemize}
    \item Kształt spirali logarytmicznej w funkcji kąta $\beta$
    \item Porównanie z Direct Pursuit -- długość trajektorii, czas przechwycenia
    \item Wizualizacja dla $\beta \in \{0°, 15°, 30°, 45°\}$
\end{itemize}

\subsection{Pościg 2D -- Proportional Navigation}

\textbf{TODO:} Analiza numeryczna:
\begin{itemize}
    \item Wpływ stałej nawigacji $N$ na trajektorię
    \item Porównanie efektywności: $N \in \{1, 2, 3, 4, 5\}$
    \item Stabilność metody dla różnych kroków czasowych
\end{itemize}

\subsection{Pościg cykliczny}

\textbf{TODO:} Analiza numeryczna:
\begin{itemize}
    \item Zbieżność do punktu centralnego dla $n \in \{3, 4, 6, 12, 18\}$ agentów
    \item Czas zbieżności w funkcji liczby agentów
    \item Kształt trajektorii zbiorczej
\end{itemize}

\subsection{Pościg 3D -- Helix Target}

\textbf{TODO:} Analiza numeryczna:
\begin{itemize}
    \item Wpływ parametrów helisy ($r$, $\omega$, $v_z$) na trajektorię ścigającego
    \item Wizualizacja 3D trajektorii
    \item Czas przechwycenia w funkcji prędkości pionowej $v_z$
\end{itemize}

\subsection{Pościg 3D -- Lissajous Target}

\textbf{TODO:} Analiza numeryczna:
\begin{itemize}
    \item Wpływ stosunku częstotliwości $\omega_x : \omega_y : \omega_z$ na kształt krzywej
    \item Wizualizacja trajektorii dla różnych kombinacji parametrów
    \item Złożoność trajektorii ścigającego
\end{itemize}

\subsection{Pościg n-wymiarowy}

\textbf{TODO:} Analiza numeryczna:
\begin{itemize}
    \item Porównanie czasu przechwycenia dla $n \in \{2, 3, 4, 5, 10, 20\}$ wymiarów
    \item Długość trajektorii w funkcji wymiaru
    \item Wpływ "przekleństwa wymiarowości" na efektywność pościgu
\end{itemize}

\subsection{Pościg na sferze $S^2$}

\textbf{TODO:} Analiza numeryczna:
\begin{itemize}
    \item Trajektorie geodezyjne na powierzchni sfery
    \item Porównanie z pościgiem w $\mathbb{R}^3$ (projekcja)
    \item Wizualizacja 3D z renderowaniem sfery
\end{itemize}

\subsection{Pościg na torusie}

\textbf{TODO:} Analiza numeryczna:
\begin{itemize}
    \item Wpływ stosunku promieni $R/r$ na trajektorię
    \item Efekt topologii toroidalnej (periodyczność)
    \item Wizualizacja 3D z renderowaniem torusa
\end{itemize}

\section{Wnioski}

\textbf{TODO:} Podsumowanie wyników:
\begin{itemize}
    \item Porównanie efektywności strategii (Direct, Constant Bearing, Proportional Navigation)
    \item Wnioski dotyczące zastosowań praktycznych
    \item Wpływ wymiaru przestrzeni na złożoność problemu
    \item Obserwacje dotyczące pościgu na rozmaitościach riemannowskich
\end{itemize}

\section*{Literatura}

\begin{enumerate}
    \item Wikipedia, \textit{Pursuit curve}. \\
    \url{https://en.wikipedia.org/wiki/Pursuit_curve}

    \item Wikipedia, \textit{Proportional navigation}. \\
    \url{https://en.wikipedia.org/wiki/Proportional_navigation}

    \item Nahin, P. J. (2007). \textit{Chases and Escapes: The Mathematics of Pursuit and Evasion}. Princeton University Press.

    \item Virtanen, P., et al. (2020). SciPy 1.0: Fundamental Algorithms for Scientific Computing in Python. \textit{Nature Methods}, 17, 261–272. \\
    \url{https://doi.org/10.1038/s41592-019-0686-2}

    \item Harris, C. R., et al. (2020). Array programming with NumPy. \textit{Nature}, 585, 357–362. \\
    \url{https://doi.org/10.1038/s41586-020-2649-2}

    \item do Carmo, M. P. (1992). \textit{Riemannian Geometry}. Birkhäuser.

    \item Shneydor, N. A. (1998). \textit{Missile Guidance and Pursuit: Kinematics, Dynamics and Control}. Horwood Publishing.

    \item Richardson, D. W. (1979). Cyclic Pursuit and Its Applications. \textit{Journal of Optimization Theory and Applications}, 29(1), 101-123.
\end{enumerate}

\end{document}

