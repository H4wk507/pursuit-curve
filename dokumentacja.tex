\documentclass[12pt,a4paper]{article}

\usepackage[utf8]{inputenc}
\usepackage[polish]{babel}
\usepackage[T1]{fontenc}
\usepackage{amsmath}
\usepackage{amsfonts}
\usepackage{graphicx}
\usepackage{geometry}
\usepackage{hyperref}
\usepackage{float}
\usepackage{listings}
\usepackage{xcolor}
\usepackage{booktabs}

\geometry{
    left=2.5cm,
    right=2.5cm,
    top=2.5cm,
    bottom=2.5cm
}

\lstset{
    basicstyle=\ttfamily\small,
    breaklines=true,
    frame=single,
    commentstyle=\color{gray},
    keywordstyle=\color{blue},
    stringstyle=\color{red}
}

\begin{document}

\begin{flushright}
Gliwice, \today
\end{flushright}

\vspace{1cm}

\begin{center}
{\Large \textbf{Modelowanie matematyczne}}\\[0.5cm]
{\Large Dokumentacja projektu końcowego}\\[1cm]

{\large \textbf{Tytuł:} Krzywa pogoni}\\[0.5cm]

{\large \textbf{Autorzy:} Skowroński Piotr, Czuba Krzysztof}\\[0.3cm]

{\large \textbf{Kierunek:} Informatyka, studia 2 stopnia (sem. II)}
\end{center}

\vspace{1cm}

\tableofcontents
\newpage

\section{Cel zadania}

Celem projektu jest implementacja i analiza różnych strategii pościgu (pursuit curves) w przestrzeni 2D, 3D oraz n-wymiarowej. Projekt obejmuje:

\begin{itemize}
    \item Implementację klasycznych strategii pościgu: Direct Pursuit, Constant Bearing, Proportional Navigation
    \item Rozwiązanie numeryczne równań różniczkowych opisujących ruch ścigającego i celu
    \item Analizę zachowania strategii dla różnych trajektorii celu (linia prosta, okrąg, helisa, krzywe Lissajous)
    \item Uogólnienie problemu na przestrzeń n-wymiarową i badanie wpływu liczby wymiarów
    \item Wizualizację trajektorii w 2D i 3D oraz analizę numeryczną dla wyższych wymiarów
\end{itemize}

\section{Opis teoretyczny}

\subsection{Wprowadzenie do problemu pościgu}

Problem pościgu (pursuit problem) polega na określeniu trajektorii ścigającego (pursuer), który dąży do przechwycenia poruszającego się celu (target). W zależności od zastosowanej strategii, ścigający może celować bezpośrednio w aktualną pozycję celu (Direct Pursuit), utrzymywać stały kąt namiarowania (Constant Bearing), lub stosować bardziej zaawansowane metody jak nawigacja proporcjonalna (Proportional Navigation).

Problemy pościgu mają szerokie zastosowania praktyczne:
\begin{itemize}
    \item Nawigacja rakiet i pocisków naprowadzanych
    \item Systemy autonomiczne (drony, roboty)
    \item Nawigacja morska i lotnicza
    \item Gry komputerowe (AI przeciwników)
    \item Optymalizacja trajektorii
\end{itemize}

\subsection{Klasyczna krzywa pościgu (Direct Pursuit)}

\subsubsection{Model matematyczny 2D}

W najprostszym przypadku Direct Pursuit, ścigający w każdej chwili celuje bezpośrednio w aktualną pozycję celu. Stan układu w przestrzeni dwuwymiarowej $\mathbb{R}^2$ opisany jest wektorami:

\begin{itemize}
    \item $\mathbf{p}(t) = [p_x(t), p_y(t)]^T$ -- pozycja ścigającego
    \item $\mathbf{t}(t) = [t_x(t), t_y(t)]^T$ -- pozycja celu
\end{itemize}

Równania różniczkowe opisujące ruch ścigającego w strategii Direct Pursuit mają postać:

\begin{equation}
\frac{d\mathbf{p}}{dt} = v_p \cdot \frac{\mathbf{t} - \mathbf{p}}{\|\mathbf{t} - \mathbf{p}\|}
\label{eq:direct_pursuit_2d}
\end{equation}

gdzie:
\begin{itemize}
    \item $v_p > 0$ -- prędkość skalarna ścigającego [m/s]
    \item $\mathbf{t} - \mathbf{p}$ -- wektor skierowany od ścigającego do celu
    \item $\|\mathbf{t} - \mathbf{p}\| = \sqrt{(t_x - p_x)^2 + (t_y - p_y)^2}$ -- odległość euklidesowa
\end{itemize}

Ruch celu opisany jest niezależnym równaniem:
\begin{equation}
\frac{d\mathbf{t}}{dt} = \mathbf{v}_t(t)
\label{eq:target_motion}
\end{equation}

gdzie $\mathbf{v}_t(t)$ jest zadaną funkcją prędkości celu.

TODO:
TODO: jakies tabelki dla różnych strategii, róznych parametrów, róznych wymiarów

\section*{Literatura}

\begin{enumerate}
    \item Wikipedia, \textit{Krzywa pogoni}. \\
    \url{https://en.wikipedia.org/wiki/Pursuit_curve}
\end{enumerate}

\end{document}

